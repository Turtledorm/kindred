\section{Descrição dos Use-Cases}


\subsection{Nome do Use-Case}
    Criação e realização de uma partida do jogo Kindred.


\subsection{Resumo}
    Entrada do usuário numa sala para jogar uma partida de Kindred
    e a realização da mesma.


\subsection{Atores}
        Jogador cliente e hospedeiro (primários).


\subsection{Precondição}
    Jogo já está aberto no menu principal.


\subsection{Fluxo Normal dos Eventos}
    \begin{enumerate}
        \item \label{escolhePartida} O usuário escolhe jogar uma partida.
        \item \label{decideHospedar} O usuário decide hospedar uma partida.
        \item O usuário escolhe as opções da partida (mapa, etc.).
        \item O usuário hospedeiro aguarda um outro jogador cliente se conectar.
        \item \label{salaConfirmada} Os usuários hospedeiro e conectado confirmam a partida.
        \item Os jogadores posicionam suas unidades no mapa para a batalha começar.
        \item \label{turnoJogadorA} Turno do jogador A. Este pode mover e atacar com suas unidades.
        \item O jogador A encerra seu turno.
        \item Turno do jogador B. Este pode mover e atacar com suas unidades.
        \item O jogador B encerra seu turno.
        \item Turno do jogador A novamente (retorna-se ao passo \ref{turnoJogadorA}).
        \item \label{exercitoDestruido} Eventualmente, um jogador destruirá todas as unidades do oponente.
    \end{enumerate}


\subsection{Fluxos Alternativos}

    \subsubsection{Fechamento do Jogo}
        No passo \ref{escolhePartida}, o usuário pode decidir sair do jogo, o que encerra o
        programa.

    \subsubsection{Fechamento do Jogo}
        Em vez de hospedar uma partida, o jogador pode optar, no passo \ref{decideHospedar},
        por se juntar a uma sala existente:
        \begin{enumerate}
            \item O usuário procura alguma sala pela qual tenha interessante nas existentes.
            \item O usuário conecta-se à sala.
            \item Retorno ao passo \ref{salaConfirmada} do fluxo normal.
        \end{enumerate}

    \subsubsection{Cancelamento da Partida}
        No passo 5, se algum dos jogadores (hospedeiro ou conectado) recusar
        a partida, retorna-se ao passo \ref{escolhePartida} do fluxo normal.


\subsection{Condições Posteriores}
    \subsubsection{Fim do Jogo}
        Quando o passo \ref{exercitoDestruido} ocorrer, a declaração do jogador vencedor é feita
        e a partida é finalizada, retornando os usuários ao passo \ref{escolhePartida}
        do fluxo normal.
