\section{Identificação dos objetos/classes necessários ao projeto
     (apenas um esboço inicial).}

\let\itemm\undefined
% bold and with colon
\newcommand{\itemm}[1]{\item\textbf{#1: }}


\includegraphics{class-diagram.1}




\begin{itemise}
    \itemm{Classe} Contém a classe à qual a \emph{unidade} pertence e muda seus
        atributos de acordo.
    \itemm{Arma} Contém a arma que a unidade está usando. Muda a range da unidade
        e o dano dela.
    \itemm{Unidade} Contém os atributos (\emph{HP}, \emph{att}, \emph{def},
        \emph{agi}, \emph{mov}) de uma \emph{unidade} do exército do jogador, bem
        como a \emph{arma} que ela está usando e a \emph{classe} à qual pertence.
    \itemm{Tile} Guarda até uma \emph{unidade} e o \emph{terrain} daquela parte do
        \emph{mapa}, e afeta a performance da \emph{unidade} (se houver uma) de
        acordo.
    \itemm{Mapa} Vai guardar todas as \emph{tiles} e suas posições.
    \itemm{Online} Responsável por cuidar da conexão, enviando e recebendo os dados
        dos jogadores.
    \itemm{Jogador} Cuida de todas as ações que o jogador pode exercer, como mover
        unidade, atacar, etc.
    \itemm{Jogo} A classe central do sistema. Cuidará das atualizações das
        demais classes, manter o loop do jogo, inicializar as demais classes e
        encerrar o jogo se o jogador o fechar. Também guarda o turno atual.
 
\end{itemise}