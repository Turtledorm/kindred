\section{Identificação dos objetos/classes necessários ao projeto
     (esboço inicial)}

\let\itemm\undefined
% bold and with colon
\newcommand{\itemm}[1]{\item\textbf{#1: }}

\begin{centre}
\includegraphics{class-diagram.1}
\end{centre}



\begin{itemise}
    \itemm{Classe} Contém a \emph{classe} à qual a \emph{unidade} pertence e muda seus
        atributos de acordo.
    \itemm{Arma} Contém a \emph{arma} que a \emph{unidade} está usando. Muda a
        \emph{range} da \emph{unidade} e o \emph{dano} dela.
    \itemm{Unidade} Contém os atributos (\emph{HP}, \emph{atq}, \emph{def},
        \emph{agi}, \emph{mov}) de uma \emph{unidade} do \emph{exército} do jogador,
        bem como a \emph{arma} que ela está usando e a \emph{classe} à qual pertence.
    \itemm{Tile} Guarda até uma \emph{unidade} e o \emph{terreno}, e afeta a performance
        da \emph{unidade} (se houver uma) de acordo.
    \itemm{Mapa} Armazena todos os \emph{tiles} e suas posições (ou seja, como estão
        relacionados entre si).
    \itemm{Online} Responsável por cuidar da conexão pela Internet, enviando e
        recebendo dados dos jogadores.
    \itemm{Jogador} Cuida de todas as ações que o jogador pode exercer, como mover
        \emph{unidade}, atacar, etc.
    \itemm{Jogo} A classe central do sistema. Cuidará das atualizações das
        demais classes, manter o loop principal do jogo, inicializar as demais classes
        e encerrar o programa se o jogador o fechar. Também guarda o \emph{turno} atual.
\end{itemise}
