\section{Dicionário de Dados}


\let\itemm\undefined
% bold and with colon
\newcommand{\itemm}[1]{\item\textbf{#1: }}

\begin{itemise}
    \itemm{Exército} Conjunto de \emph{unidades} controladas por um jogador.
    \itemm{Unidade} Uma personagem do \emph{exército} do jogador. Possui \emph{hp},
        \emph{atq}, \emph{def}, \emph{agi}, \emph{mov} e uma \emph{classe}, e pode
        equipar uma \emph{arma}. Se um jogador não possuir nenhuma \emph{unidade}, ele
        perde a partida.
    \itemm{Turno} Rodada de um jogador. Nela, a pessoa de quem é a vez pode mover suas
        \emph{unidades} e atacar o \emph{exército} do oponente enquanto este assiste o 
        que ocorre. Ao encerrar o \emph{turno}, passa-se a vez para o adversário jogar.
    \itemm{Classe} Grupo à qual uma \emph{unidade} pertence. Afeta o \emph{HP}, \emph{mov},
        \emph{atq}, \emph{def}, \emph{agi} e limita as armas usadas pela \emph{unidade}.
    \itemm{HP} Hit Points. A quantidade de dano que uma \emph{unidade} pode receber
        antes de ser destruída.
    \itemm{Atq} Representação numérica do dano que uma \emph{unidade} pode
        causar.
    \itemm{Def} Representação numérica da redução do dano sofrido de uma
        \emph{unidade}.
    \itemm{Agi} Representação numérica da agilidade de uma \emph{unidade}. Afeta a
        chance de se esquivar de um ataque.
    \itemm{Mov} Representa o número de \emph{tiles} que uma \emph{unidade} pode se
        deslocar a cada turno, horizontalmente e verticalmente.
    \itemm{Range} Alcance do ataque de uma unidade. Uma \emph{unidade} pode atacar
        um oponente que esteja até um número \emph{range} de \emph{tiles} à distância.
        O valor varia de acordo com a \emph{arma} equipada.
    \itemm{Arma} Arma que uma unidade pode equipar. Cada tipo de arma tem seu alcance.
        Pode afetar o \emph{atq} da \emph{unidade} de acordo com o \emph{range} do ataque.
    \itemm{Dano} Representa o \emph{HP} perdido por uma \emph{unidade} que recebeu um
        ataque. O cálculo do \emph{dano} envolve o \emph{atq} do atacante e os atributos
        \emph{def} e \emph{agi} do defensor.
    \itemm{Tile} É um \"quadrado\" que compõe o \emph{mapa}. Cada \emph{tile} tem um
        \emph{terrain} e pode conter, no máximo, uma \emph{unidade}.
    \itemm{Terreno} Espécie de \"ambiente\" que ocorre em um \emph{tile} (por exemplo,
        floresta ou planície). Pode afetar o \emph{mov}, \emph{def} ou \emph{agi},
        dando benefícios ou prejuízos à \emph{unidade} que está no \emph{tile}.
    \itemm{Mapa} Um conjunto de \emph{tiles} no qual os jogadores controlam seus
        \emph{exércitos} e se enfrentam.
\end{itemise}
