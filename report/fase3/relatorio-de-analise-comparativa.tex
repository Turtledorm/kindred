\section{Análise comparativa}

Infelizmente, devido aos problemas apresentados no relatório de erros encontrados
durante os testes, a grande maioria dos diagramas que havíamos feito se tornaram
ultrapassados. No entanto, como refatoramos o código pensando nos testes a serem
feitos, acabamos tendo o trabalho simplificado na hora de gerar os testes.

Se não houvesse a necessidade de refatorar o código, não há dúvidas de que os
diagramas que já havíamos gerado serviriam de base para boa parte dos testes, e
esses diagramas nos ajudariam muito no decorrer do desenvolvimento de tais testes.
Fazer testes para \emph{use-cases} em que não há diagramas é algo mais complicado,
visto que é preciso ter uma ótima ideia do funcionamento da implementação a ser
testada.

Em suma, não podemos fazer uma comparação entre a geração de testes para os quais
haviam diagramas e para os quais não haviam, no entanto é do entendimento do grupo
que os diagramas com certeza facilitariam o trabalho.
