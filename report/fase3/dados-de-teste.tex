\section{Dados de teste}

\newcommand{\teste}[3]{
    \item\textbf{#1:}
    \begin{itemise}
        \item\textbf{Entrada: } #2
        \item\textbf{Saída: }   #3
    \end{itemise}
}

\begin{itemise}

\teste
    {testClient\_WrongIP}
    {Um client é criado e manipulado para se conectar a um servidor inexistente:
    um que possui o nome\emph{fakehost}.}
    {A saída esperada seria ocorrer um \texttt{RuntimeException}, mostrando que
    houve erro na conexão com o servidor.}

\teste
    {testClient\_CorrectIP}
    {Um client é criado e manipulado para se conectar a um servidor já previamente
    rodando no \emph{localhost}.}
    {A saída esperada seria não ocorrer um \texttt{RuntimeException}, visto que a
    conexão foi feita com sucesso.}

\teste
    {testQuit}
    {É criado e inicializado um Client, que logo envia um comando para sair.}
    {A saída esperada seria receber uma afirmação positiva do \texttt{assertFalse},
    ou seja, é falso que o Client esteja conectado após o comando \texttt{QUIT}.}

\teste
    {testQuit}
    {É criado e inicializado um Client, que logo envia um comando para sair.}
    {A saída esperada seria receber uma afirmação positiva do \texttt{assertFalse},
    ou seja, é falso que o Client esteja conectado após o comando \texttt{QUIT}.}
    
\teste
    {testNick\_Invalid}
    {Cliente envia \texttt{NICK} (comando de terminal) com os seguintes parâmetros,
    todos inválidos:
        \begin{itemise}
            \item \texttt{aa}
            \item \texttt{3444}
            \item \texttt{abc\%}
        \end{itemise}
    }
    {Espera-se que o \emph{nickname} do cliente não mude em nenhum dos casos. Ou
    seja, que \texttt{client.getNickname()} seja \texttt{null} (valor para
    \emph{nickname} indefinido) em todos os casos testados.}

\teste
    {testNick\_Valid}
    {Cliente envia \texttt{NICK} com o seguinte parâmetro: \texttt{niceNick}, um
    parâmetro válido.}
    {Esperamos que o Cliente de fato recebe e guarde a \emph{string}
    \texttt{niceNick} como seu \emph{nickname}, pois ela é válida.}


\teste
    {testHost\_Invalid}
    {Um Cliente vira \emph{host} de uma partida. Temos três situações disponíveis:
        \begin{enumerate}
            \item Cliente sem \emph{nickname} usando um mapa que não existe. No caso,
                os parâmetros são: \texttt{null} (\emph{nickname}) e
            "\texttt{thisIsFake}" (nome do mapa).
            \item Cliente sem \emph{nickname} usando um mapa válido. No caso, os
                parâmetros são: \texttt{null} e "\texttt{simpleMap}".
            \item Cliente com \emph{nickname} válido usando um mapa inválido. No
                caso, os parâmetros são: "\texttt{hostInvalid}" (um \emph{nickname}
                válido) e "\texttt{thisIsFake}".
        \end{enumerate}
    }
    {Espera-se que os testes nos digam que:
        \begin{itemise}
            \item Em 1, \emph{nickname} e nome de mapa de fato não valem.
            \item Em 2, \emph{nickname} não vale, mas nome do mapa vale.
            \item Em 3, \emph{nickname} vale, mas nome do mapa não.
        \end{itemise}
    }
    
\teste
    {testHost\_Valid}
    {Cliente com \emph{nickname} válido vira \emph{host} de uma partida usando um
    mapa válido. Após um tempo, ele muda para um outro mapa válido. Os parâmetros
    usados são: "\texttt{hostValid}" (\emph{nickname} do \emph{host}),
    "\texttt{simpleMap}" e "\texttt{happyPlains}" (nome dos mapas válidos).}
    {Espera-se que o Cliente de fato sirva de \emph{host} para os mapas escolhidos,
    mesmo após a mudança.}

\teste
    {testUnhost\_Invalid}
    {Há dois casos:
        \begin{enumerate}
            \item Cliente com nickname inválido e sem ser um \emph{host} (não deu o
            comando ou não escolheu um mapa válido) tenta dar \texttt{UNHOST}.
            Neste caso, os parâmetros são nulos.
            \item Cliente com nickname válido e sem ser um \emph{host} tenta dar
            \texttt{UNHOST}. Neste caso, os parâmetros acabam sendo nulos, já que
            não há sala.
        \end{enumerate}
    }
    {Espera-se que, por o Cliente não ser \emph{host}, o comando \texttt{UNHOST} não
    funcione.}

\teste
    {testeHost\_Valid}
    {Cliente com nickname válido começa a ser \emph{host} de uma sala usando um mapa
    válido. Após um tempo, ele dá o comando \emph{unhost}. Os parâmetros são:
    "\texttt{unhostVal}" (nickname) e "\texttt{testmap}".}
    {Espera-se que o Cliente possa se tornar \emph{host} e cancelar o comando sem
    problemas.}

\teste
    {testJoin\_Valid}
    {Dois Cliente são conectados ao servidor com nicknames válidos. Um será o
    \emph{host} e o outro o \emph{guest}. O \emph{host} cria uma sala usando um mapa
    válido e o \emph{guest} entra nela. Os parâmetros são: "\texttt{hostJoin}" e
    "\texttt{guestJoin}" (nomes do \emph{host} e do \emph{guest}) e
    "\texttt{testmap}" (nome do mapa). Em seguida, eles desconectam.}
    {Espera-se que o servidor reconheça que os dois Clientes estão em jogo antes de
    eles se desconectarem.}

A partir daqui, temos os testes para partidas em si. Usaram-se mapas especialmente preparados para testes,  \emph{testmap} (para quase todos os casos) e \emph{testmatch} (para testar o fim da partida), além de uma unidade de debug chamada \emph{Master}.

\teste
    {testSurrender}
    {Dois Clientes, um \emph{host} e um \emph{guest}, se conectam ao servidor e uma
    sala é criada. O jogo começa e o \emph{host} (primeiro a jogar) dá o comando
    \texttt{SURRENDER} para desistir da partida. Logo após os dois se desconectam.
    Os parâmetros são "\texttt{hostSur}" e "\texttt{guestSur}" como nicknames do
    \emph{host} e do \emph{guest} e "\texttt{testmap}" como mapa.}
    {Espera-se que os dois Clientes sejam reconhecidos como "\texttt{isPlaying}"
    até o momento em que o \emph{host} dá o comando \texttt{SURRENDER}. Na
    averiguação subsequente, espera-se que nem o \emph{host} nem o \emph{guest}
    sejam reconhecidos como "\texttt{isPlaying}".}

\teste
    {testEndTurn}
    {Um Cliente é criado e vira um \emph{host}. É verificado o turno (-1). Então um
    \emph{guest} entra na sala e o jogo começa. É verificado o turno (1-turno do
    \emph{host}). O \emph{host} dá o comando endTurn. É verificado o turno (2-turno
    do \emph{guest}). O \emph{guest} dá o comando endTurn. É verificado o turno
    (1-turno do \emph{host}). Os parâmetros são "\texttt{hostEnd}",
    "\texttt{guestEnd}" e "\texttt{testmap}". Além dos já mencionados -1, 1 e 2 para
    as comparações de turno.}
    {Espera-se que de fato os turnos mudem da forma descrita acima.}

\teste
    {testMove\_Valid}
    {Um \emph{host} e um \emph{guest} estão em jogo. O \emph{host} manda a unidade na
    localização 0 0 se mover para a localização 1 0. Após uma pausa ele tenta
    movê-la de novo.}
    {Espera-se que de fato o \emph{host} consiga mover a unidade da primeira vez,
    mas que ele falhe na segunda tentativa.}

\teste
    {testMove\_Invalid}
    {Durante uma partida, um \emph{host} tenta mover uma unidade usando diversos
    comandos inválidos, seja tentando mover uma unidade em uma posição inválida,
    seja tentando colocar uma unidade em uma posição inválida, ou partindo de uma
    posição inválida para tentar posicionar em outra posição inválida. Também é
    testado posicionar uma unidade em uma tile ocupada, ou posicionar em um lugar
    que extrapole o movimento da unidade selecionada, assim como tentar mover
    uma unidade inimiga.}
    {Espera-se que nenhum dos comandos testados sejam aceitos, já que são todos
    inválidos.}

\teste
    {testAttack\_Valid}
    {Durante uma partida, o \emph{host} comanda uma unidade para atacar um inimigo.
    É dada uma pausa e o \emph{host} tenta mover e atacar com a mesma unidade
    novamente. Os parâmetros são "\texttt{hostAtkVl}", "\texttt{guestAtkVl}" e
    "\texttt{testmap}".}
    {Espera-se que a unidade selecionada ataque com sucesso na primeira vez.
    Em seguida, é esperado que o \emph{host} falha em tentar movimentar ela e
    atacar com ela.}

\teste
    {testAttack\_Invalid}
    {Durante uma partida, o \emph{host} tenta atacar. São testadas diversas
    coordenadas inválidas, tiles vazias como sendo alvo, atacar uma unidade além do
    range da unidade sendo usada, atacar usando uma unidade inimiga e atacar uma
    unidade aliada. Os parâmetros são "\texttt{hostAtkIn}", "\texttt{guestAtkIn}" e
    "\texttt{testmap}".}
    {Espera-se que nenhuma das entradas testadas seja bem-sucedida, já que são
    inválidas.}

\teste
    {testUnitsNextTurn}
    {Durante uma partida, um \emph{host} movimenta uma unidade e ataca uma unidade
    inimiga, encerrando seu turno em seguida. O \emph{guest} então encerra seu turno,
    passando a vez para o \emph{host}, que tenta mover uma unidade e atacar usando
    ela. Os parâmetros são "\texttt{hostCtrl}", "\texttt{guestCtrl}" e
    "\texttt{testmap}".}
    {Espera-se que o \emph{host} possa usar suas unidades normalmente após o
    \emph{guest} encerrar seu turno.}

\teste
    {testMatch}
    {Durante uma partida, o \emph{host} destrói todas as unidades adversárias, efetivamente vencendo o oponente.}
    {Espera-se que tanto o \emph{host} quanto o \emph{guest} automaticamente saiam da partida e voltem para o menu principal, onde podem interagir com o servidor e procurar novos oponentes. Curiosamente, o programa passa neste teste, mas notamos que os jogadores ficam travados na partida após o ataque final quando experimentamos com o projeto.}


\end{itemise}
