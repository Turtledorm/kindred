\section{Problemas no design e na implementação}

Como abordado nos demais relatórios, o grupo precisou refatorar toda a parte do
código responsável por gerenciar as trocas entre cliente e servidor, pois fazer
testes com o código implementado da maneira que estava antes era algo extremamente
complicado. Sendo assim, o grupo agiu para se adequar aos requisitos da Entrega 3. 

Tal refatoração basicamente anulou boa parte dos diagramas de design que havíamos
feito, embora, a um alto nível, alguns ainda pudessem ser utilizados. O jogo em si
era funcional, mas a necessidade de implementar testes falou mais alto.
 
Tendo em vista que há uma seção específica no enunciado para abordar esse assunto,
é claro que algo assim era esperado, o que, sinceramente, não é surpreendente. Foi
a primeira vez que o grupo implementou testes automatizados, e o programa começou
a ser feito antes de nos aprofundarmos nesse assunto em questão durante as aulas.
O grupo todo sabe da importância dos testes e que escrever um programa pensando nos
testes a serem feitos facilita o trabalho em diversos sentidos. É uma lição que
não será esquecida.
