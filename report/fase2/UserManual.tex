\documentclass{article}
% uncomment to set the page format to A4 instead of A5
\def\RELEASEDOCUMENT{}

% language and encoding
\usepackage[utf8]{inputenc}
\usepackage[T1]{fontenc}
\usepackage[brazil]{babel}

% fonts: normal and monospaced
\usepackage[charter]{mathdesign}
%\usepackage{mathpazo}
%\usepackage{newtxtext, newtxmath}
%\usepackage[scaled]{ulgothic}
\usepackage{inconsolata}

% page format
\ifdefined\RELEASEDOCUMENT
\usepackage[landscape, a4paper, margin=2cm]{geometry}
\else
\usepackage[landscape, a5paper, margin=0.7cm]{geometry}
\fi

% additional commands to math mode
\usepackage{amsmath}

% indent first line of first paragraph
\usepackage{indentfirst}

% page numbering
\pagestyle{empty}

% automatically format quotes (no need to write as ``'')
\usepackage{csquotes}
\MakeOuterQuote{"}


% support coloured links, PDF bookmarks, etc.
%\usepackage[colorlinks=false, pdfstartview=FitH, pdfpagelayout=OneColumn]{hyperref}

% remove space after a comma when it acts like a decimal separator
\usepackage{icomma}

% "cancel" terms with a slash
%\usepackage{cancel}

% enable use of colours
%\usepackage{xcolor}

% show customised enumerate 
%\usepackage{enumerate}

% support graphics
\usepackage{graphicx}

% input raw text
%\usepackage{verbatim}

% add cells that span more than one row or column
%\usepackage{multirow, multicol}

% include external raw PDF pages
%\usepackage{pdfpages}

% allow forcing force a figure to be displayed "here"
%\usepackage{float}

% input code
%\usepackage{listings} \lstset{basicstyle=\ttfamily}

% add "lorem ipsum" text
%\usepackage{blindtext} \blindmathtrue

% add questions and answers
\usepackage[]{exercise}
\renewcommand{\ExerciseName}{Questão}
\renewcommand{\ExerciseListName}{Q\!\!}
\renewcommand{\AnswerName}{Resposta}
\renewcommand{\AnswerListName}{Resposta}
\renewcommand{\ExePartName}{Item}
\renewcommand{\ExerciseHeader}{\hrule~\par\noindent{\textbf{\large
            \ExerciseName~\ExerciseHeaderNB\ExerciseHeaderTitle
            \ExerciseHeaderOrigin. \medskip}}}
\renewcommand{\ExePartHeader}{~\par\noindent{\textit{\large
            (\ExePartHeaderNB)
            \smallskip}}}
\renewcommand{\theExePart}{\alph{ExePart}}
\renewcommand{\ExePartHeaderNB}{\theExePart}



% add Portuguese trig functions
\DeclareMathOperator{\sen}{sen}
\DeclareMathOperator{\tg}{tg}
\DeclareMathOperator{\cotg}{cotg}
\DeclareMathOperator{\cossec}{cossec}
\DeclareMathOperator{\arcsen}{arcsen}
\DeclareMathOperator{\arctg}{arctg}
\DeclareMathOperator{\arccotg}{arccotg}
\DeclareMathOperator{\arccossec}{arccossec}

% add British commands and environments
\newenvironment{centre}[0]{\begin{center}}{\end{center}}
\newenvironment{itemise}[0]{\begin{itemize}}{\end{itemize}}


\title{
    Kindred -- Manual do Usuário
}

\author{
    Bruno Guilherme Ricci Lucas (4460596)\\
    Leonardo Pereira Macedo (8536065)\\
    Vinícius Bitencourt Matos (8536221)
}

\begin{document}

\maketitle

\section{Introdução}

\emph{Kindred} é um jogo de estratégia multiplayer online jogado por turnos, baseado em jogos como Advance Wars e Fire Emblem. A ideia de turnos é que apenas um usuário joga por vez, enquanto seu adversário assiste suas ações. Posteriormente, o controle é passado ao oponente quando o usuário decreta o fim de seu turno, e assim segue o ritmo do jogo. \par

As partidas são individuais, com um jogador (nome dado ao usuário do sistema) enfrentando apenas um adversário em um mapa escolhido. Cada jogador tem à sua disposição um exército: seus soldados são genericamente chamados de \emph{unidades}. O objetivo do jogo é eliminar as tropas adversárias em sua totalidade, levando seu exército à vitória.

\section{Pré-Requisitos}
Kindred atualmente é um jogo feito apenas para plataformas Unix. Para jogá-lo, o usuário deve ter instalado:
    \begin{enumerate}
        \item Java 7 (ou maior);
        \item Ant: necessário para compilar o jogo.
    \end{enumerate}

\section{Como Executar}
    \begin{enumerate}
    \item Abra um terminal e vá até o diretório do jogo.
    \item Digite, no root do projeto: \par
            \texttt{\$ ant build} \par
            Com isso, cria-se um diretório \emph{bin/} onde o jogo pode ser executado.
    \item Para executar o Kindred, execute algum dos comandos abaixo no terminal. Se tiver alguma dúvida, digite no root do projeto: \par
            \texttt{\$ ./run.sh -h} \par
        \begin{enumerate}
        \item Servidor: \par
                \texttt{\$ ./run.sh -s}
        \item Cliente: \par
                \texttt{\$ ./run.sh -c}
        \end{enumerate}
    \end{enumerate}

\section{Modo de Uso}
    Atualmente, o jogo só suporta uma CLI (interface por linha de comando), ou seja, toda a interação do usuário com o servidor/cliente será feita pelo terminal. Abaixo vão mais instruções de como interagir com o programa:
    \subsection{Servidor}
        O usuário não tem como interagir diretamente com o servidor. O único comando aceito é \emph{CLOSE}, responsável por finalizar e fechá-lo com segurança;
    \subsection{Cliente (fora da partida)}
        Ao abrir o cliente, pede-se o endereço IP do servidor para o cliente. Importante: o programa não funciona se não houver conexão com algum servidor! \par
        Feito isso, o usuário entrará no menu principal do jogo, onde alguns comandos digitados, juntamente com seus argumentos, podem ser utilizados. O caractere de espaço \texttt{' '} deve ser usado para separar comandos e argumentos. \par
        Os comandos aceitos pelo servidor, juntamente com seus argumentos, são:
        \begin{enumerate}
        \item \textbf{NICK} [nome] \par
            Se o parâmetro nome for dado, define o nome do usuário no servidor. Caso contrário, apenas mostra ao usuário o nome que definiu no servidor. \par

            O nickname definido deve ter de 3 a 10 caracteres alfanuméricos, sendo que o primeiro caractere deve ser obrigatoriamente uma letra. \par

            Retorna uma mensagem de erro se o nome já estiver sendo usado por outro usuário ou se for inválido (isto é, fora da definição especificada).
        \item \textbf{MAPS} \par
            Mostra uma lista de mapas na qual se pode jogar. Mais detalhes sobre mapas são dados na seção \emph{Regras do Jogo}.
        \item \textbf{ROOMS} \par
            Mostra uma lista de todas as salas existentes. Uma sala é um local no servidor onde um usuário pode esperar um outro usuário qualquer se conectar para poderem disputar uma partida num mapa já definido. \par
            A lista de salas é composta pelo nome do usuário que criou a sala e o mapa escolhido para jogar uma partida.
        \item \textbf{HOST} [NOME DO MAPA] \par
            Cria uma sala para jogar no mapa especificado. Cada usuário pode criar no máximo uma sala; se outra for criada, a anterior é desfeita. \par
            \emph{Atenção:} Este comando só funciona depois que o usuário registrou um nome no servidor! \par
            Retorna uma mensagem de erro se o nome do mapa especificado não for encontrado.
        \item \textbf{UNHOST} \par
            Desfaz a sala criada pelo usuário, se ela existir. Caso contrário, nada ocorre. \par
            \emph{Atenção:} Este comando só funciona depois que o usuário registrou um nome no servidor!
        \item \textbf{JOIN} [NOME] \par
            Conecta o usuário na sala criada pelo usuário de nome especificado. \par
            \emph{Atenção:} Este comando só funciona depois que o usuário registrou um nome no servidor! \par
            Retorna uma mensagem de erro se não existir uma sala criada pelo referido nome.
        \item \textbf{HELP} \par
            Indica como interagir com o servidor ao mostrar esta mesma lista de comandos para o usuário.
        \item \textbf{QUIT} \par
            Disconecta o usuário do servidor. \par
            Como sinônimo, também pode ser usado o comando \textbf{EXIT}.
        \end{enumerate}
     \subsection{Cliente (dentro da partida)}
        Após entrar numa sala, ambos os jogadores são notificados e o jogo começa! Os comandos suportados agora são diferentes. A estrutura para mandar mensagens é a mesma de fora da partida: através de comandos e argumentos, separados pelo caractere de espaço \texttt{' '}. \par

        Para maiores detalhes técnicos de como jogar e o que significam os termos do jogo, consulte a seção \emph{Regras do Jogo}. \par

        A interação com o jogo em si é feita através dos seguintes comandos, juntamente com seus argumentos:
        \begin{enumerate}
        \item \textbf{MOVE} [xi] [yi] [xf] [yf]\par
            Move a \emph{unidade} localizada na posição (xi, yi) do mapa para a posição (xf, yf). \par
            Retorna uma mensagem de erro se algo impede o movimento (\emph{tile} ocupado, coordenadas inválidas etc.). \par
            Como sinônimo, também pode ser usado o comando \textbf{MV}.
        \item \textbf{ATTACK} [xi] [yi] [xf] [yf]\par
            Faz a \emph{unidade} na posição (xi, yi) do mapa atacar a unidade na posição (xf, yf). \par
            Retorna uma mensagem de erro se algo impede o ataque (coordenadas inválidas, \emph{unidade} alvo fora do alcance etc.). \par
            Como sinônimo, também pode ser usado o comando \textbf{ATK}.
        \item \textbf{INFO} [x] [y]\par
            Dá informações sobre o \emph{Tile} na posição (x, y), detalhando qual seu \emph{terreno} e \emph{unidade}, se houver.
        \item \textbf{HELP} \par
            Indica como agir no jogo ao mostrar esta mesma lista de comandos para o usuário.
        \item \textbf{END} \par
            Encerra o turno do usuário, passando a vez para o adversário.
        \item \textbf{SURRENDER} \par
            Faz com que o usuário desista do jogo, encerrando a partida e concedendo a vitória ao adversário.
        \end{enumerate}

\section{Regras do Jogo}

\subsection{Unidades}
Cada jogador começa com um certo número de \emph{unidades}, e cada \emph{unidade} tem suas características expressas em valores numéricos denominados \emph{atributos}. Estes são melhor explicados na Tabela 1. Uma utilização inteligente do mapa e das características das \emph{unidades} em jogo, tanto aliadas quanto adversárias, vão definir o vencedor da partida. Estratégia é essencial para dominar este jogo. \par

\emph{Unidades} pertencentes a jogadores diferentes podem batalhar entre si. O \emph{HP} da \emph{unidade} defensora é reduzido de acordo com seus \emph{atributos} e os da \emph{unidade} atacante.

\begin{table}[H]
\centering
\begin{tabular}{|c|l|p{0.7\textwidth}|}\hline
 \textbf{Sigla} & \textbf{Nome} & \textbf{Descrição} \\\hline
 MOV & Movimento & Número máximo de tiles (horizontais e verticais) de que uma unidade pode se deslocar durante o turno. \\\hline
 RNG & Alcance   & Distância máxima, em tiles (horizontais e verticais) que uma unidade tem para atacar uma unidade adversária. \\\hline
 HP  & HP        & Vida da unidade. Quando chegar a zero, a unidade é destruída. \\\hline
 ATK & Ataque    & Quantidade de dano que uma unidade pode causar em uma batalha. \\\hline
 DEF & Defesa    & Redução de dano aplicada quando uma unidade sofre um ataque. \\\hline
 AGI & Agilidade & Chance de acertar e de se esquivar do ataque de uma unidade adversária. \\\hline
\end{tabular}
\caption{Atributos de uma unidade}
\end{table}

\subsection{Classes}
Cada \emph{unidade} pertence a uma \emph{classe} que define seus \emph{atributos}; ou seja, todas as \emph{unidades} pertencentes a uma \emph{classe} são iguais. As \emph{classes} disponíveis e suas características estão indicadas na Tabela 2. Veja que cada \emph{classe} conta com sua gama de forças e fraquezas, definidas não só pelos \emph{atributos} mas também pelo uso que se dá a elas. \par

Um \emph{Wizard}, por exemplo, tem o maior \emph{alcance} do jogo e um \emph{ataque} bem elevado, mas também a menor \emph{defesa} e um baixo \emph{movimento}. Ou seja, o \emph{Wizard} será letal se usado corretamente à distância e se estiver sempre bem posicionado. No entanto, por sua inerente fragilidade, um erro de cálculo pode custar-lhe a vida e, consequentemente, o jogo. Manter seu \emph{Wizard} afastado do núcleo de combate e protegido por outras \emph{unidades} é recomendado.

\begin{table}[H]
\centering
\begin{tabular}{|l|c|c|c|c|c|c|c|c|}\hline
 \textbf{Nome} & \textbf{Símbolo} & \textbf{HP} & \textbf{MOV} & \textbf{RNG} & \textbf{ATK} & \textbf{DEF} & \textbf{AGI}\\\hline
Archer    & A & 15 & 4 & 3 & 10 &  5 &  9 \\\hline
Knight    & K & 30 & 5 & 1 &  6 & 12 &  6 \\\hline
Lancer    & L & 25 & 5 & 2 &  8 &  8 &  7 \\\hline
Rogue     & R & 18 & 7 & 1 &  6 &  5 & 15 \\\hline
Swordsman & S & 20 & 6 & 1 &  7 &  7 & 12 \\\hline
Wizard    & W & 10 & 4 & 4 & 12 &  3 &  8 \\\hline
\end{tabular}
\caption{Classes existentes}
\end{table}

\subsection{Mapas}
Os mapas têm tamanhos variáveis e são divididos em "quadradinhos" menores chamados \emph{tiles}, que podem ser ocupados por apenas uma \emph{unidade}. Cada \emph{tile} representa um tipo de \emph{terreno}, que podem ou não atribuir alguns bônus para a \emph{unidade} que nela se encontra. Estes aprimoramentos incluem um aumento em \emph{defesa} e/ou \emph{agilidade}. No entanto, quase todos os \emph{terrenos} também afetam a mobilidade de suas tropas. Portanto, fique atento para onde posiciona suas \emph{unidades}! Verifique a Tabela 3 para saber os terrenos existentes.

\begin{table}[H]
\centering
\begin{tabular}{|l|l|c|c|c|}\hline
 \textbf{Nome} & \textbf{Cor} & \textbf{Bônus de defesa} & \textbf{Bônus de agilidade} & \textbf{Penalidade de movimento}\\\hline
 Road     & Cinza claro  &   0   &   0   &  0 \\\hline
 Plains   & Marrom claro &   0   & +10\% & -1 \\\hline
 Forest   & Verde        & +10\% &   0   & -1 \\\hline
 Swamp    & Roxo         & +20\% & -10\% & -2 \\\hline
 Mountain & Preto        & +30\% &   0   & -3 \\\hline
\end{tabular}
\caption{Terrenos existentes}
\end{table}

\subsection{Estratégias}
O primeiro passo antes de qualquer batalha é conhecer seu exército. Dominando suas \emph{unidades}, o jogador já tem meio caminho andado para a vitória. O passo seguinte é observar o \emph{mapa}. Ver que tipo de \emph{terreno} predomina e se existe algum foco de um \emph{terreno} distinto do predominante lhe permite formular as estratégias necessárias para conduzir as ações do inimigo a seu favor e assegurar uma vitória. \par

Em um mapa pequeno, por exemplo, as \emph{unidades} de longo alcance -- como \emph{Archers} e \emph{Wizards} -- têm vantagem, pois estarão próximas de aliados que as podem defender e os oponentes não terão muito espaço para manobrar ou fugir. Por que não, então, despachar unidades com alta mobilidade para eliminar essas \emph{unidades} inimigas antes que elas causem estrago? Da mesma forma, você poderia também usar essas mesmas \emph{unidades} como isca, enquanto prepara uma elaborada armadilha para seu oponente. \par

Após o conhecimento de suas \emph{unidades} e do \emph{mapa}, sua adaptabilidade e engenhosidade serão os fatores que desequilibrarão uma partida!

\end{document}
